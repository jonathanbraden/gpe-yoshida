\documentclass{revtex4}
\usepackage{amsmath,amssymb}

\newcommand{\abs}[1]{\ensuremath{\left|#1\right|}}

\begin{document}
Here we define the set of units used in our numerical code and show how the dimensionless equations of motion are adjusted as we change the units.

For simplicity, we will consider the equation of motion for a single BEC (i.e.\ the nonlinear Schrodinger equation).  The derivations are essentially identical in the multispecies case, so we do not produce them here in the interests of simplicity.

Our starting point is the nonlinear Schrodinger equation describing the evolution of a BEC condensate in the dilute gas limit
\begin{equation}
  i\hbar\dot{\psi} = -\frac{\hbar^2}{2m}\nabla^2\psi + V\psi + \frac{4\pi\hbar^2 a_s}{m}(N-1)\abs{\psi}^2 \psi \, .
\end{equation}
We assume the reader is familiar with the derivation of this approximation.
Here $N$ is the number of particles, $m$ is the particle mass, and $a_s$ is the S-wave scattering length for $2\to 2$ particle scattering, and $V$ is the external trapping potential.  For now, this equation is for a three-dimensional condensate.
The condensate wavefunction is normalized as
\begin{equation}
  \int d^3x\abs{\psi}^2 = 1 \, ,
\end{equation}
so $\psi$ has units of $L^{-3}$ (i.e.\ it is a density).
With this normalization, $N\abs{\psi}^2$ is interpreted as the local particle number density.
A common alternative notation is to use
\begin{equation}
  g = \frac{4\pi\hbar^2a_s}{m} \, .
\end{equation}
The interpretation of $g$ in this case comes from approximating the interparticle interaction potential
\begin{equation}
  V({\bf r}-{\bf r'}) \approx g\delta({\bf r}-{\bf r'}) \, .
\end{equation}
Dusting off our undergraduate scattering theory, we know that the corresponding S-wave scattering length is related to $g$ in the Born approximation as
\begin{equation}
  a_s = \frac{m}{4\pi\hbar^2}\int V({\bf r}) d^3r
\end{equation}
in three dimensions, thus producing the relationship between $g$ and $a_s$.

We now want to put this equation into dimensionless form.
We introduce a characteristic frequency $\omega_p$, inverse length $\kappa_p$ and wavefunction scale $\sqrt{\Lambda}$ and define
\begin{equation}
  \bar{t} = \omega_p t \qquad \bar{x} = \kappa_p x \qquad \bar{\psi} = \frac{\psi}{\sqrt{\Lambda}} \, .
\end{equation}
where $\bar{\cdot}$ indicates a dimensionless variable.
The equation of motion therefore becomes
\begin{equation}
  i\frac{\partial\bar{\psi}}{\partial\bar{t}} = -\frac{1}{2}\frac{(\hbar\kappa_p)^2}{m\hbar\omega_p}\bar{\nabla}^2\bar{\psi} + \frac{V}{\hbar\omega_p}\bar{\psi} + \frac{4\pi\hbar^2 a_s}{\hbar\omega_p m}(N-1)\Lambda^2\abs{\bar{\psi}}^2\bar{\psi} \, ,
\end{equation}
with normalization
\begin{equation}  
  \int d^3x \abs{\bar{\psi}}^2 = \Lambda^{-1} \, .
\end{equation}

We therefore introduce dimensionless parameters
\begin{equation}
  \alpha = \frac{\hbar^2\kappa_p^2}{m\hbar\omega_p} \qquad \bar{V} = \frac{V}{\hbar\omega_p} \qquad \bar{g} = \frac{4\pi\hbar^2a_s}{\hbar\omega_p m}(N-1)\Lambda^2
\end{equation}
and rewrite
\begin{equation}
    i\frac{\partial\bar{\psi}}{\partial\bar{t}} = -\frac{\alpha}{2}\bar{\nabla}^2\bar{\psi} + \bar{V}\bar{\psi} + \bar{g}\abs{\bar{\psi}}^2\bar{\psi} \, .
\end{equation}
As a quick sanity check, it is straightforward to see that all of our parameters are indeed dimensionless.

\section{Reduction to Lower Dimensions}
The parameters above are defined in three-dimensions.  While the dimensionality is irrelevant for the ``linear'' parameters (i.e.\ everything by $\bar{g}$), the interpretation of $\bar{g}$ is dimension dependent, as should be clear from the dependence of the dimension of $\Lambda$ on the number of spatial dimensions.

\end{document}
